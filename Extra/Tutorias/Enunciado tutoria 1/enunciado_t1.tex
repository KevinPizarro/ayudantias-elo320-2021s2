\documentclass[letterpaper,10pt]{article}
\usepackage[utf8]{inputenc}
\usepackage[spanish]{babel}

\usepackage{fullpage}
\usepackage{hyperref}

%opening
\title{Ejercicio 1 - Estructuras de Datos y Algoritmos}
\author{}

\begin{document}

\maketitle

El ejercicio siguiente constará de 6 pasos. Idealmente le debería tomar unos 15
minutos realizar cada uno.

Los contenidos a tratar serán los siguientes:
\begin{itemize}
 \item \textsc{Tipos:} char, int, float, strings, booleanos, overflow, arreglos.
 \item \textsc{Operadores:} diferenciar ``='' y ``=='', aritméticos, operador módulo, incrementar/decrementar, precedencia.
 \item Creación de \textsc{makefiles}.
 \item Búsqueda y utilización de las funciones de \textit{gnu libc}. (Uso de documentación)
 \item \textsc{Control de flujo:} condicionales e iteraciones.
 \item \textsc{Funciones}, empleando sólo paso por valor.
 \item Lectura y escritura de \textsc{valores en consola} (funciones printf y scanf)
 \item \textsc{Uso de la terminal}, para compilar, ejecutar y depurar.
 \item Directivas de preprocesador. Uso de \textsc{\#include}.
\end{itemize}

\paragraph{Consideraciones previas al trabajo}
\begin{itemize}
 \item El código deberá estar escrito según el estándar de codificación GNU.
 \item El código escrito debe estar perfectamente indentado con tabuladores, no espacios.
 \item Algunos archivos de documentación no fueron cargados a la máquina virtual.
       En esta guía hay enlaces a esos documentos faltantes. Si los necesita, descárquelos y déjelos en el escritorio de la máquina virtual.
\end{itemize}

\paragraph{Recomendaciones}
\begin{itemize}
 \item Gran parte del código de esta tutoría se ejecutará dentro de un bloque
  \textsc{while}. Trate de dividir su código en funciones de forma que este sea fácil de leer. Todo lo que necesita saber sobre funciones se encuentra en el punto 5 del manual de C que se encuentra disponible en el escritorio de su máquina virtual \textit{(gnu-c-manual.pdf.)}
\end{itemize}

\newpage
\paragraph{Guía}
\begin{enumerate}
 \item Escriba la estructura básica de un programa en C. Si no la conoce lea el punto 5.7 de gnu-c-manual.pdf. Luego escriba un mensaje de
 bienvenida utilizando la función \textsc{printf} (libc.pdf pag. 253 +\\ \href{http://goo.gl/0xrGh}{http://goo.gl/0xrGh} 2.1).
 Finalmente compile el programa que acaba de escribir
 con \textsc{gcc} y ejecutelo utilizando \textsc{ddd}. Tenga en cuenta que para poder ejecutar el programa con \textsc{ddd} es necesario
 compilar el programa con información de debugging. Si no entiende algo de la última instrucción consulte los manuales del sistema con el
 comando \textsc{man}. Si aún tiene dudas o no sabe usar el terminal de Linux consulte al ayudante.

 \item Se desea simular el funcionamiento de un ascensor. Defina un arreglo global (\textit{gnu-c-manual.pdf} 2.5) de tipo \textsc{char} que represente la
 cantidad de pasajeros esperando al ascensor en cada piso del edificio. El largo del arreglo representará la cantidad de pisos del edificio,
 codifíquelo de modo que se pueda adaptar el código a una cantidad arbitraria modificando sólo una constante (\href{http://goo.gl/0xrGh}{http://goo.gl/0xrGh} 3.1). Luego
 modifique su función \textsc{main} para que después de imprimir el mensaje de bienvenida se entre en un bucle que incremente aleatoriamente la cantidad
 de pasajeros cada 5 segundos. Ayúdese de \textit{libc.pdf} para encontrar funciones que le permitan lograr esto. Además le servirá utilizar la sentencia
 while explicada en el punto 4.5 de \textit{gnu-c-manual.pdf}. ¿que pasa si hay 1000 pasajeros en un piso?

 \item Defina dos variables globales que representen la cantidad de pasajeros que están dentro del ascensor y el piso en que el ascensor se encuentra.
 Ambas variables deben comenzar con valor cero. Escriba un algoritmo para que el ascensor pase a buscar a los pasajeros a los pisos correspondientes.
 El ascensor debe poder moverse solo un piso cada 5 segundos! Además el ascensor soporta un máximo de 8 pasajeros.

 \item Cree un arreglo para almacenar los pisos que han sido marcados en el ascensor. ¿De qué tipo debe ser el arreglo? Luego escriba un algoritmo
 que simule la elección del piso hecha por los pasajeros al momento de subir al ascensor. Modifique el algoritmo del punto 3 para que los pasajeros
 sean dejados en el piso que corresponde. Asegure que ningún pasajero se mantenga indefinidamente dentro del ascensor.

 \item Escriba una función que en cada iteración del bloque \textsc{while} (cada 5 segundos) muestre por pantalla el piso actual en que se encuentra
 el ascensor, la cantidad de personas que hay en el ascensor, el peso total de las personas que hay en el ascensor, la cantidad de personas
 que hay esperando en cada piso y los pisos que estan marcados en la botonera del ascensor. ¡Ojo, las personas pesan 73,327542 Kilos cada una!

 \item Comente su código, escriba un \textsc{makefile} (\href{http://goo.gl/F8qCi}{http://goo.gl/F8qCi}) y un \textit{readme.txt}. Estos tres elementos facilitan la lectura, compilación y
 comprensión del código que está escribiendo.
En el archivo \textit{readme.txt} debe especificarse que es cada archivo de su programa, como debe ser compilado y como debe ser ejecutado. Además debe
 explicar cualquier particularidad que pueda tener su programa.
\end{enumerate}

\paragraph{Nota!}
Al finalizar la tutoría pida a su ayudante que revise su trabajo. Luego limpie los archivos que haya creado en su máquina virtual.
Asegúrese solucionar todas sus dudas antes de irse!


\end{document}
