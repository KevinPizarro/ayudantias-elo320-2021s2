\documentclass[a4paper,10pt]{article}
\usepackage[utf8x]{inputenc}
\usepackage[spanish]{babel}
\usepackage{amsmath}
\usepackage{graphicx}
\usepackage{fullpage}

\newif\ifpdf
\ifx\pdfoutput\undefined
   \pdffalse
\else
   \pdfoutput=1
   \pdftrue
\fi
\ifpdf
   \usepackage{graphicx}
   \usepackage{epstopdf}
   \DeclareGraphicsRule{.eps}{pdf}{.pdf}{`epstopdf #1}
   \pdfcompresslevel=9
\else
   \usepackage{graphicx}
\fi

%opening
\title{Tutoría 5 - Estructuras de datos y algoritmos}
\author{Cristóbal Ganter y Felipe Vera}
\date{8 de octubre de 2012}

\begin{document}

\maketitle

\paragraph{¡Aviso!} ¡Recuerde dejar la máquina virtual en orden una vez que termine su trabajo!
¡Pregunte al ayudante apenas tenga una duda!

Si tiene problemas de \textit{segmentation fault} o \textit{stack overflow} en su programa, una buena opción es \textit{debuggear} su programa con el comando \texttt{ddd nombre-de-programa}.
Estos errores son difíciles de encontrar, por lo que llenar su programa de \texttt{printf}s no lo ayudará mucho.

Lea la guía completa antes de empezar a trabajar. Esto le permitirá desarrollar los programas teniendo en cuenta lo que viene más adelante.

No se espera que los alumnos sepan usar las funciones \texttt{fread} y \texttt{fwrite}. Si tiene problemas, lea la documentación en \textit{libc.pdf} y luego pregunte a su ayudante.

\paragraph{Contenido de esta semana}
\begin{itemize}
  \item punteros, punteros dobles y operador \&
  \item Asignación dinámica de memoria: stack, heap, \texttt{malloc} y \texttt{free}.
  \item Estructuras de datos dinámicas: árboles binarios de búsqueda. Creación y lectura.
  \item Empleo de funciones recursivas para los contenidos anteriores.
	\item Empleo de funciones para guardar, leer y obtener información de archivos binarios y de texto.
\end{itemize}

\paragraph{Actividad}
\begin{enumerate}
	\item \textbf{Creación de un árbol binario de búsqueda}
	El árbol binario de búsqueda, es un método efectivo para ordenar elementos de una lista y acceder a ellos usando memoria dinámica.
	En este caso, se desea ordenar un grupo de personas según su edad.

	\begin{enumerate}
		\item Cree una estructura que permita construir un árbol binario de búsqueda. La estructura además debe poder guardar el nombre y la edad de una persona.
		\item En su función \texttt{main} permita al usuario ingresar nombres de personas y sus respectivas edades. Cada par nombre edad debe guardarse en una estructura
			y cada estructura debe insertarse en un árbol binario de búsqueda (ABB). \textit{El ABB debe estar ordenado según edad.}
	\end{enumerate}

	\item \textbf{Acceso a los ítemes del árbol binario. Recorrer el árbol en orden ascendente}
	Mediante una función recursiva, se puede obtener todos los nodos de árbol de búsqueda de forma ordenada.

	\begin{enumerate}
		\item Escriba una función recursiva que recorra el ABB ascendentemente imprimiendo el \textbf{nombre} de cada persona.
		\item Escriba una función idéntica a la anterior que, en vez de imprimir cada nodo, lo guarde en un archivo binario utilizando la función \texttt{fwrite}.
		\item Llame a la función del punto 2.b. desde su función \texttt{main}.
	\end{enumerate}

	\item \textbf{Manejo de archivos: Cargar el árbol guardado}
	
	\begin{enumerate}
		\item Reserve suficiente memoria como para cargar al archivo generado en el punto 2. Puede utilizar la función \texttt{stat} para encontrar el tamaño del archivo.
		\item Lea el archivo generado en el punto 2 utilizando la función \texttt{fread}. Los datos leídos desde el archivo deben ser almacenados en el espacio de memoria reservado en el punto anterior.
		\item Utilizando la función escrita en 2.a. intente de recorrer el ABB. ¿que ha pasado? ¿cual es el problema? ¿qué alternativas de solución se le ocurren?
	\end{enumerate}


\end{enumerate}

\end{document}
