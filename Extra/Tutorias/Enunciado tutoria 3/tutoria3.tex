\documentclass[a4paper,10pt]{article}
\usepackage[utf8x]{inputenc}
\usepackage[spanish]{babel}
\usepackage{amsmath}
\usepackage{fullpage}

%opening
\title{Tutoría 3 - Estructuras de datos y algoritmos}
\author{Cristóbal Ganter y Felipe Vera}

\begin{document}

\maketitle

\paragraph{¡Aviso!} ¡Recuerde dejar la máquina virtual en orden una vez que termine su trabajo!
¡Pregunte al ayudante apenas tenga una duda!

Si tiene problemas de \textit{segmentation fault} o \textit{stack overflow} en su programa, una buena opción es \textit{debuggear} su programa con el comando \texttt{ddd nombre-de-programa}.
Estos errores son difíciles de encontrar, por lo que llenar su programa de \texttt{printf}s no lo ayudará mucho.

\paragraph{Contenido de esta semana}
\begin{itemize}
  \item punteros, punteros dobles y operador \&
  \item Asignación dinámica de memoria: stack, heap, \texttt{malloc} y \texttt{free}.
  \item Estructuras de datos dinámicas: listas doblemente enlazadas, inserción en la cola, inserción en la posición especificada, búsqueda.
  \item Empleo de funciones recursivas para los contenidos anteriores.
\end{itemize}

\paragraph{Actividad}
\begin{enumerate}
  \item Listas doblemente enlazadas.
  \begin{enumerate}
    \item Se desea modelar la fila frente a la caja de un banco. Cree una estructura llamada \texttt{persona} que represente a una persona en la fila. La estructura debe permitir construir
	  una lista doblemente enlazada, con los atributos \texttt{struct persona *prev, *next}, su \texttt{edad} y la \textbf{cantidad de \texttt{dinero} que traen}.
    \item En su función \texttt{main} cree un puntero a la primera persona de la fila, esto es la persona que está más cerca de la caja. ¿Hacia donde apuntará este
	  puntero si aún no hemos creado personas?
    \item Escriba una función que le permita crear personas. Las personas deben ser creadas con edad aleatoria entre 18 y 200 años y cantidad de dinero aleatoria entre
	  0 y $2^{32}$ unidades monetarias. ¿De qué tipo tiene que ser la variable usada para almacenar la cantidad de dinero?
  \end{enumerate}
  \item Inserción al final de la lista
  \begin{enumerate}
    \item Escriba una función \texttt{void ins\_persona (persona **final, persona *nuevo\_elemento)} que le permita insertar personas al final de la fila.
    \item En su función \texttt{main} cree un un bucle \texttt{while} que se ejecute una vez cada 10 segundos. Si tiene problemas utilizando la función
	  \texttt{sleep} avise inmediatamente a su ayudante!
    \item Al comenzar el bucle \texttt{while} debe eliminarse a la primera persona de la fila. Esto simulará al usuario que acaba de ser atendido y sale de la fila.
	  La memoria que ocupa este usuario debe ser liberada.
    \item Después de quitar al primer usuario deben agregarse aleatoriamente entre 0 y 3 nuevos usuarios al final de la fila.
  \end{enumerate}
  \item Cantidad de elementos en una lista.
  \begin{enumerate}
    \item Para cada iteración del bucle \texttt{while} imprima por pantalla todos los usuarios que se encuentran en la fila indicando su edad y cantidad de dinero.
	  Además imprima la cantidad total de personas esperando en la fila.
  \end{enumerate}
  \item Búsqueda en una lista
  \begin{enumerate}
    \item Para cada iteración del bucle while busque a todas las personas mayores de 70 años y permita que pasen al comienzo de la fila. No es necesario que las personas
	  que reasigne al principio de la fila sigan un orden determinado. Esta búsqueda debe ser realizada desde la persona que está más cerca de la caja a la que
	  está más lejos.
  \end{enumerate}
  \item Inserción en una lista
  \begin{enumerate}
    \item Suponga que su fila tampoco está libre de gente que quiera “colarse”. Para simular este comportamiento, cree una función \texttt{void insertar(persona ** fila, persona * interpuesta, int posicion)}.
	  Tenga precaución con los casos en que fila o interpuesta sean nulos, y la posición (contada desde el principio de la fila) exceda los límites de esta fila.
    \item Incluya esta función en el bucle \texttt{while} que hizo en la parte 2. La probabilidad de que una persona se “cuele” en la fila debe ser del 25\%,
	  aleatoriamente en cualquier posición de la fila. Sea cuidadoso de no generar \textit{segmentation faults}.
  \end{enumerate}
  \item Función recursiva
  \begin{enumerate}
    \item Una de las personas, al ser atendida, tiene problemas de dinero y el resto de las personas en la fila son inusualmente bondadosas. Cree una función recursiva
	  con prototipo \texttt{int bondad(persona *siguiente)}. Esta función recibe como argumento una persona y retorna dinero. Al ser llamada se ejecutan tres
	  acciones:
    \begin{itemize}
      \item La persona que ha sido pasada como argumento ejecuta la función bondad sobre la persona que está inmediatamente detrás de ella.
      \item Una vez que retorne la función \texttt{bondad} aplicada a la persona de atrás, el retorno de la función debe ser sumado al dinero de la persona que se
	    ha pasado por argumento.
      \item La persona que se ha pasado como argumento le entrega un 10\% de su dinero a quien llamó la función. Esto es, el 10\% del dinero
	    de la persona es usado como retorno de la función.
    \end{itemize}
    \item La función \texttt{bondad} debe ser llamada por primera vez con la segunda persona (desde la caja) como argumento y el dinero de salida debe ser asignado
	  a la primera persona.

  \end{enumerate}




\end{enumerate}


\end{document}
